% В этом файле следует писать текст работы, разбивая его на
% разделы (section), подразделы (subsection) и, если нужно,
% главы (chapter).

% Предварительно следует указать необходимую информацию
% в файле SETUP.tex

%% В этот файл не предполагается вносить изменения

% В этом файле следует указать информацию о себе
% и выполняемой работе.

\documentclass [fontsize=14pt, paper=a4, pagesize, DIV=calc]%
{scrartcl}
% ВНИМАНИЕ! Для использования глав поменять
% scrartcl на scrreprt

% Здесь ничего не менять
\usepackage [T2A] {fontenc}   % Кириллица в PDF файле
\usepackage [utf8] {inputenc} % Кодировка текста: utf-8
\usepackage [russian] {babel} % Переносы, лигатуры

%%%%%%%%%%%%%%%%%%%%%%%%%%%%%%%%%%%%%%%%%%%%%%%%%%%%%%%%%%%%%%%%%%%%%%%%
% Создание макроса управления элементами, специфичными
% для вида работы (курс., бак., маг.)
% Здесь ничего не менять:
\usepackage{ifthen}
\newcounter{worktype}
\newcommand{\typeOfWork}[1]
{
	\setcounter{worktype}{#1}
}

% ВНИМАНИЕ!
% Укажите тип работы: 0 - курсовая, 1 - бак., 2 - маг.,
% 3 - бакалаврская с главами.
\typeOfWork{0}
% Считается, что курсовая и бак. бьются на разделы (section) и
% подразделы (subsection), а маг. — на главы (chapter), разделы и
%  подразделы. Если хочется,
% чтобы бак. была с главами (например, если она большая),
% надо выбрать опцию 3.

% Если при выборе 2 или 3 вы забудете поменять класс
% документа на scrreprt (см. выше, в самом начале),
% то получите ошибку:
% ./aux/appearance.tex:52: Package scrbase Error: unknown option ` chapterprefix=

%%%%%%%%%%%%%%%%%%%%%%%%%%%%%%%%%%%%%%%%%%%%%%%%%%%%%%%%%%%%%%%%%%%%%%%%
% Информация об авторе и работе для титульной страницы

\usepackage {titling}

% Имя автора в именительном падеже (для маг.)
\newcommand {\me}{%
Степаненко Сергей Денисович%
}

% Научный руководитель
\newcommand{\supervisorA}%
{старший преподаватель В. Н. Брагилевский}

% Научный руководитель
\newcommand{\supervisorB}%
{М. Сохацкий}

% Год публикации
\date{2019}

% Название работы
\title{Изоморфизм расслоенного пространства и Пи-типа}

% Кафедра
%
\newcommand{\department}{Кафедра информатики и вычислительного эксперимента}

\newcommand {\direction} {%
по направлению 02.\ifthenelse{\value{worktype} = 2}{04}{03}.02 ---
Фундаментальная информатика\\и информационные технологии%
}

%%%%%%%%%%%%%%%%%%%%%%%%%%%%%%%%%%%%%%%%%%%%%%%%%%%%%%%%%%%%%%%%%%%%%%%%
% Другие настраиваемые элементы текста

% Листинги с исходным кодом программ: укажите язык программирования
\usepackage{listings}
\lstset{
    language=[ISO]C++,%  Язык указать здесь
    basicstyle=\small\ttfamily,
    breaklines=true,%
    showstringspaces=false%
    inputencoding=utf8x%
}
% полный список языков, поддерживаемых данным пакетом, есть,
% например, здесь (стр. 13):
% ftp://ftp.tex.ac.uk/tex-archive/macros/latex/contrib/listings/listings.pdf

% Нумерация списков: можно при необходимести
% изменять вид нумерации (например, добавлять правую скобку).
% По умолчанию буду списки вида:
% 1.
% 2.
% Изменять вид нумерации можно в начале нумерации:
% \begin{enumerate}[1)] (В квадратных скобках указан желаемый вид)
\usepackage[shortlabels]{enumitem}
                    \setlist[enumerate, 1]{1.}

% Гиперссылки: настройте внешний вид ссылок
\usepackage%
[pdftex,unicode,pdfborder={0 0 0},draft=false,%backref=page,
    hidelinks, % убрать, если хочется видеть ссылки: это
               % удобно в PDF файле, но не должно появиться на печати
    bookmarks=true,bookmarksnumbered=false,bookmarksopen=false]%
{hyperref}


\usepackage {amsmath}      % Больше математики
\usepackage {amssymb}
\usepackage {textcase}     % Преобразование к верхнему регистру
\usepackage {indentfirst}  % Красная строка первого абзаца в разделе
\usepackage{bussproofs}
\usepackage{tikz}
\usetikzlibrary{positioning}

\usepackage {fancyvrb}     % Листинги: определяем своё окружение Verb
\DefineVerbatimEnvironment% с уменьшенным шрифтом
	{Verb}{Verbatim}
	{fontsize=\small}

% Вставка рисунков
\usepackage {graphicx}

% Графики
\usepackage{pgfplots}

% Commutative diagrams
\usepackage{pb-diagram}

% Общее оформление
% ----------------------------------------------------------------
% Настройка внешнего вида

%%% Шрифты

% если закомментировать всё — консервативная гарнитура Computer Modern
\usepackage{paratype} % профессиональные свободные шрифты
%\usepackage {droid}  % неплохие свободные шрифты от Google
%\usepackage{mathptmx}
%\usepackage {mmasym}
%\usepackage {psfonts}
%\usepackage{lmodern}
%var1: lh additions for bold concrete fonts
%\usepackage{lh-t2axccr}
%var2: the package below could be covered with fd-files
%\usepackage{lh-t2accr}
%\usepackage {pscyr}

% Геометрия текста

\usepackage{setspace}       % Межстрочный интервал
\onehalfspacing

\newlength\MyIndent
\setlength\MyIndent{1.25cm}
\setlength{\parindent}{\MyIndent} % Абзацный отступ
\frenchspacing            % Отключение лишних отступов после точек
\KOMAoptions{%
    DIV=calc,         % Пересчёт геометрии
    numbers=endperiod % точки после номеров разделов
}

                            % Консервативный вариант:
%\usepackage                % ручное задание геометрии
%[%                         % (не рекомендуется в проф. типографии)
%  margin = 2.5cm,
  %includefoot,
  %footskip = 1cm
%] %
%  {geometry}

%%% Заголовки


\ifthenelse{{\value{worktype} > 1}}{%
  \KOMAoptions{%
      headings=normal,   % размеры заголовков поменьше стандартных
      chapterprefix=true,% Печатать слово Глава
      appendixprefix=true% Печатать слово Приложение
  }
}{% Печатать слово Приложение даже если нет глав
  \newcommand*{\appendixmore}{%
    \renewcommand*{\sectionformat}{%
    \appendixname~\thesection\autodot\enskip}
    \renewcommand*{\sectionmarkformat}{%
      \appendixname~\thesection\autodot\enskip}
  }
}

% шрифт для оформления глав и названия содержания
\newcommand{\SuperFont}{\Large\sffamily\bfseries}

% Заголовок главы
\ifthenelse{\value{worktype} > 1}{%
\renewcommand{\SuperFont}{\Large\normalfont\sffamily}
\newcommand{\CentSuperFont}{\centering\SuperFont}
\usepackage{fncychap}
\ChNameVar{\SuperFont}
\ChNumVar{\CentSuperFont}
\ChTitleVar{\CentSuperFont}
\ChNameUpperCase
\ChTitleUpperCase
}

% Заголовок (под)раздела с абзацного отступа
\addtokomafont{sectioning}{\hspace{\MyIndent}}

\renewcommand*{\captionformat}{~ ~}
\renewcommand*{\figureformat}{Figure~\thefigure}

% Плавающие листинги
\usepackage{float}
\floatstyle{ruled}
\floatname{ListingEnv}{Listing}
\newfloat{ListingEnv}{htbp}{lol}[section]

% точка после номера листинга
\makeatletter
\renewcommand\floatc@ruled[2]{{\@fs@cfont #1.} #2\par}
\makeatother


%%% Оглавление
\usepackage{tocloft}

% шрифт и положение заголовка
\ifthenelse{\value{worktype} > 1}{%
\renewcommand{\cfttoctitlefont}{\hfil\SuperFont\MakeUppercase}
}{
\renewcommand{\cfttoctitlefont}{\hfil\SuperFont}
}

% слово Глава
\usepackage{calc}
\ifthenelse{\value{worktype} > 1}{%
\renewcommand{\cftchappresnum}{Глава }
\addtolength{\cftchapnumwidth}{\widthof{Глава }}
}

% Очищаем оформление названий старших элементов в оглавлении
\ifthenelse{\value{worktype} > 1}{%
\renewcommand{\cftchapfont}{}
\renewcommand{\cftchappagefont}{}
}{
\renewcommand{\cftsecfont}{}
\renewcommand{\cftsecpagefont}{}
}

% Точки после верхних элементов оглавления
\renewcommand{\cftsecdotsep}{\cftdotsep}
%\newcommand{\cftchapdotsep}{\cftdotsep}

\ifthenelse{\value{worktype} > 1}{%
    \renewcommand{\cftchapaftersnum}{.}
}{}
\renewcommand{\cftsecaftersnum}{.}
\renewcommand{\cftsubsecaftersnum}{.}
\renewcommand{\cftsubsubsecaftersnum}{.}

%%% Списки (enumitem)

\usepackage {enumitem}      % Списки с настройкой отступов
\setlist %
{ %
  leftmargin = \parindent, itemsep=.5ex, topsep=.4ex
} %

% По ГОСТу нумерация должны быть буквами: а, б...
%\makeatletter
%    \AddEnumerateCounter{\asbuk}{\@asbuk}{м)}
%\makeatother
%\renewcommand{\labelenumi}{\asbuk{enumi})}
%\renewcommand{\labelenumii}{\arabic{enumii})}

%%% Таблицы: выбрать более подходящие

\usepackage{booktabs} % считаются наиболее профессионально выполненными
%\usepackage{ltablex}
%\newcolumntype {L} {>{---}l}

%%% Библиография

\usepackage{csquotes}        % Оформление списка литературы
\usepackage[
  backend=biber,
  hyperref=auto,
  sorting=none, % сортировка в порядке встречаемости ссылок
  language=english,
  citestyle=gost-numeric,
  bibstyle=gost-numeric
]{biblatex}
\addbibresource{biblio.bib} % Файл с лит.источниками 

% Настройка величины отступа в списке
\ifthenelse{\value{worktype} < 2}{%
\defbibenvironment{bibliography}
  {\list
     {\printtext[labelnumberwidth]{%
    \printfield{prefixnumber}%
    \printfield{labelnumber}}}
     {\setlength{\labelwidth}{\labelnumberwidth}%
      \setlength{\leftmargin}{\labelwidth}%
      \setlength{\labelsep}{\dimexpr\MyIndent-\labelwidth\relax}% <----- default is \biblabelsep
      \addtolength{\leftmargin}{\labelsep}%
      \setlength{\itemsep}{\bibitemsep}%
      \setlength{\parsep}{\bibparsep}}%
      \renewcommand*{\makelabel}[1]{\hss##1}}
  {\endlist}
  {\item}
}{}

% ----------------------------------------------------------------
% Настройка переносов и разрывов страниц

\binoppenalty = 10000      % Запрет переносов строк в формулах
\relpenalty = 10000        %

\sloppy                    % Не выходить за границы бокса
%\tolerance = 400          % или более точно
\clubpenalty = 10000       % Запрет разрывов страниц после первой
\widowpenalty = 10000      % и перед предпоследней строкой абзаца

% ----------------------------


% Стили для окружений типа Определение, Теорема...
% Оформление теорем (ntheorem)

\usepackage [thmmarks, amsmath] {ntheorem}
\theorempreskipamount 0.6cm

\theoremstyle {plain} %
\theoremheaderfont {\normalfont \bfseries} %
\theorembodyfont {\slshape} %
\theoremsymbol {\ensuremath {_\Box}} %
\theoremseparator {:} %
\newtheorem {mystatement} {Утверждение} [section] %
\newtheorem {mylemma} {Лемма} [section] %
\newtheorem {mycorollary} {Следствие} [section] %

\theoremstyle {nonumberplain} %
\theoremseparator {.} %
\theoremsymbol {\ensuremath {_\diamondsuit}} %
\newtheorem {mydefinition} {Definition} %

\theoremstyle {plain} %
\theoremheaderfont {\normalfont \bfseries} 
\theorembodyfont {\normalfont} 
%\theoremsymbol {\ensuremath {_\Box}} %
\theoremseparator {.} %
\newtheorem {mytask} {Задача} [section]%
\renewcommand{\themytask}{\arabic{mytask}}

\theoremheaderfont {\scshape} %
\theorembodyfont {\upshape} %
\theoremstyle {nonumberplain} %
\theoremseparator {} %
\theoremsymbol {\rule {1ex} {1ex}} %
\newtheorem {myproof} {Доказательство} %

\theorembodyfont {\upshape} %
%\theoremindent 0.5cm
\theoremstyle {nonumberbreak} \theoremseparator {\\} %
\theoremsymbol {\ensuremath {\ast}} %
\newtheorem {myexample} {Пример} %
\newtheorem {myexamples} {Примеры} %

\theoremheaderfont {\itshape} %
\theorembodyfont {\upshape} %
\theoremstyle {nonumberplain} %
\theoremseparator {:} %
\theoremsymbol {\ensuremath {_\triangle}} %
\newtheorem {myremark} {Замечание} %
\theoremstyle {nonumberbreak} %
\newtheorem {myremarks} {Замечания} %


% Титульный лист
% Макросы настройки титульной страницы
% В этот файл не предполагается вносить изменения

%\usepackage {showframe}

% Вертикальные отступы на титульной странице
\newcommand{\vgap}{\vspace{16pt}}

% Помещение города и даты в нижний колонтитул
\usepackage{scrlayer}
\DeclareNewLayer[
  foot,
  foreground,
  contents={%
    \raisebox{\dp\strutbox}[\layerheight][0pt]{%
      \parbox[b]{\layerwidth}{\centering Ростов-на-Дону --- \thedate%
       \\\mbox{}
       }}%
  }
]{titlepage.foot.fg}
\DeclareNewPageStyleByLayers{titlepage}{titlepage.foot.fg}


\AtBeginDocument
{

  \begin{titlepage}
  
    \thispagestyle{titlepage}

    {\centering
    
    \MakeTextUppercase {МИНОБРНАУКИ России}

    \vgap

    Федеральное государственное автономное образовательное\\
    учреждение высшего образования\\
    \MakeTextUppercase {<<Южный федеральный университет>>}

    \vgap

	Институт математики, механики и компьютерных наук
    имени~И.\,И.\,Воровича
    
    \department
    
    \vspace* {\fill}
    
    \MakeTextUppercase{\me}
    
    \vspace* {\fill}
    
    {\usefont{T2A}{PTSansCaption-TLF}{m}{n}
    \MakeTextUppercase{\thetitle}}
    
    \vspace* {\fill}

	\ifthenelse{\value{worktype} = 2}{
    \MakeTextUppercase{Магистерская диссертация}\\
    }{}
    \ifthenelse{\value{worktype} = 0}{
    \MakeTextUppercase{Курсовая работа}\\
    }{}
    \ifthenelse{\value{worktype} = 1 \OR \value{worktype} = 3}{
    \MakeTextUppercase{Выпускная квалификационная работа\\
    на степень бакалавра}
	}{}
    \direction
   
    \vgap
    
    Научные руководители:\\
    \supervisorA,\\
    \supervisorB

  	\vspace {\fill}

  }\end{titlepage}
  
  \tableofcontents
  
  \clearpage
}


% Команды для использования в тексте работы


% макросы для начала введения и заключения
\newcommand{\Annot}{\addsec{Annotation}}
\ifthenelse{\value{worktype} > 1}{%
    \renewcommand{\Intro}{\addchap{Annotation}}%
}

\newcommand{\Conc}{\addsec{Сonclusion}}
\ifthenelse{\value{worktype} > 1}{%
    \renewcommand{\Conc}{\addchap{Сonclusion}}%
}

% Правильные значки для нестрогих неравенств и пустого множества
\renewcommand {\le} {\leqslant}
\renewcommand {\ge} {\geqslant}
\renewcommand {\emptyset} {\varnothing}

% N ажурное: натуральные числа
\newcommand {\N} {\ensuremath{\mathbb N}}

% значок С++ — используйте команду \cpp
\newcommand{\cpp}{%
C\nolinebreak\hspace{-.05em}%
\raisebox{.2ex}{+}\nolinebreak\hspace{-.10em}%
\raisebox{.2ex}{+}%
}

% Неразрывный дефис, который допускает перенос внутри слов,
% типа жёлто-синий: нужно писать жёлто"/синий.
\makeatletter
    \defineshorthand[russian]{"/}{\mbox{-}\bbl@allowhyphens}
\makeatother


\endinput

% Конец файла



\NewBibliographyString{langjapanese}
\NewBibliographyString{fromjapanese}

\begin{document}

\Annot

Системы верификации --- программы, разработанные для формализации доказательств на компьютере. Б\'{о}льшая часть из них базируется на теории типов, которая делится на множество подвидов.\\
Гомотопическая теория типов (Homotopy type theory, HoTT) выделяется тем, что базируется на связи между теорией типов и теорией гомотопий. Эта область лежит в основе Унивалентных оснований математики (Univalent Foundation, UF) --- попытки формализовать математику, используя в качестве фундамента не множества, а гомотопические типы или $\infty$-группоиды, а также высшие индуктивные типы (Higher Inductive Types, HIT). Этот подход интересен тем, что позволяет работать с гомотопической теорией, используя синтетический метод, то есть, не опираясь на более базовые примитивы, как, например, множества.\\
В данный момент существует несколько систем верификации, которые поддерживают гомотопическую теорию типов нативно или благодаря расширениям, но эта поддержка обеспечивается различными наборами примитивных операций, для некоторых из которых существует совсем мало примеров.\\
В работе рассматривается формализация нетривиального расслоения над окружностью --- ленты Мёбиуса, а также портирование доказательства изоморфизма тривиального расслоения и зависимого типа на верификатор доказательств Arend. Это доказательство показывает, что базовые теоретико-типовые конструкции соответствуют гомотопическим.\\
Код, которому посвящена данная работа, можно найти на GitHub\autocite{Grp1} в репозитории организации Groupoid Infinity. Доказательство выполнено в файле Fiber.ard\autocite{Fiber}, формализация ленты Мёбиуса --- в файле Moebius.ard\autocite{Moebius}.

\Introduction

Proof assistants are computer systems, which are designed to do mathematics on a computer. Some proof assistants also allow to define functions and compute on them, but their main focus is on doing proofs. They allow to define some statements and to reason about their correctness. A few of them, which are also automated theorem provers, also help a user with a set of well chosen decision procedures that allow to automatically prove formulas of a specific restricted format\autocite{ProofAssistants1}.

Large part of proof assistants are based on type theory, which is due to Curry–Howard correspondence---correspondence between logical and type-theoretic operations. There are a few classifications of type theories. One of these classifications divides type theories between two flavors: \textit{intensional type theories} and \textit{extensional type theories}. The former is the flavor in which types that represent equalities are not necessarily propositions. Type theory, which is not intensional, is called extensional. The following work was performed with one of the flavors of intensional type theory---homotopy type theory, which takes seriously the natural interpretation of identity types as formalizing path space objects in homotopy theory. This type theory comes with a few benefits: higher inductive types, which help to obtain quotients objects or free structures, univalence that simplify the process of moving back and forth between isomorphic structures, and a synthetic approach to homotopy theory.

There are several proof assistants, which support homotopy type theory natively or via extensions. Some of them are Coq\autocite{Coq}, Agda\autocite{Agda}, cubicaltt\autocite{Cubicaltt}, and redtt\autocite{Redtt}. They come with different sets of most basic operations. Some of these sets are studied better than another. Arend\autocite{Arend} is a proof assistant, which provides users with an interval type (a type that is inductively constructed from two terms, whose interpretation is as the endpoints of the interval, together with a path between them); an eliminator for an interval type---\textit{coe}, which allows, for every type over the interval, to transport elements from the fiber over left to the fiber over an arbitrary point; a dependent path type, which consist of all functions that maps elements of the interval type to the term of type parameterized by the given term of the interval type; and a function \textit{iso}, which can be used to define univalence---one of the cornerstones of homotopy type theory\autocite{Arenddocs}\autocite{nlab}.

Homotopy type theory relies on the idea that type theoretic constructions correspond to some equivalents from homotopy theory. E.g., the statement that the term $a$ is of type $A$, which is written as $a:A$, can be expressed as ``$A$ is a space and $a$ is a point of A''. In addition to the kinds of types and terms above, we also may consider types and terms with parameters. These are usually called \textit{dependent} types (or \textit{type families}) and terms (if $B$ is a type, then we might have a type $(x:B)\ E(x)$, which is parameterized by $B$). From the homotopy theoretic point of view, we might think of such a type as a \textit{fibration} $E \to B$ over the space $B$. The fibration makes precise the idea of one space (\textit{fiber}) being parameterized by another (\textit{base}) with fibers being equivalent in some coherent way. % actually homotopy equivalent

In this paper I report on formalizing a Moebius strip and on porting a proof of an equivalence between the fibration and the type family using the proof assistant Arend. The former task is a demonstration of basic Arend features and the latter is an evidence of a correspondence between basic constructions of homotopy and type theories. The code can be found in Arend Groupoid Infinity repository (moebius strip: Moebius.ard\autocite{Moebius}, proof: Fiber.ard\autocite{Fiber}).

\section{Preliminaries}

Type theory contains two basic judgements. The first, typing judgement $a : A$, states that a term $a$ has type $A$. The motivation behind this notion varies:
\begin{enumerate}
	\item $a$ is an element of set $A$.
	\item $A$ is a problem and $a$ is a solution of $A$.
	\item $a$ is a proof of a proposition $A$.
	\item $A$ is a space and $a$ is a point of $A$.
\end{enumerate}
The first perspective is due to Russell, the second is due to Kolmogorov, the third is due to Curry and Howard and the fourth alternative comes from homotopy type theory\autocite{Warren1}. 
The second judgement, $a \equiv b : A$, asserts, that terms $a$ and $b$ are judgementally equal at type $A$. Judgemental equality is a relation between linguistic expressions and can be used to rewrite expressions\autocite{hottbook}. A notion of assumption is present in type theories in form of type-context.
A definition of the type-context depends on the concrete theory. For the following definitions it can be thought of as any finite set of judgements. E.g. $\Gamma = \{a : A, b : A, c : C\}$. The turnstile $\vdash$ between a context and a type judgement should be interpreted as: ``in the context $\Gamma$, the expression $a$ has type $A$''. E.g. $\Gamma \vdash a : A$. The rules of type theory tell us how to construct new types, how to construct terms of some types, and how to use such terms to construct terms of other types. The rules can be written as follows:
\begin{prooftree}
\AxiomC{$\Gamma \vdash f : A \to B$}
\AxiomC{$\Gamma \vdash a : A$}
\BinaryInfC{$\Gamma \vdash f(a) : B$}
\end{prooftree}
This rule states that given terms $f : A \to B$ and $a : A$ in the context of $\Gamma$ we can derive a term $f(a) : B$.
Also we can construct dependent types. Dependent types can be defined using a few postulates\autocite{Wellen1}:
\begin{enumerate}
  \item A Russell style hierarchy of type universes (types whose terms are types) is given: $U_0, U_1, U_2, \dots$. The hierarchy is cumulative: $U_m : U_n$ for $m < n$, and also if $A : U_m$ and $m \leq n$, then $A : U_m$. With universes, we can write the judgement ``$A$ is a type'' as a judgement that $A$ is a term of some universe. The HoTT book uses this approach.  
  \item If the universe level is unambiguous, then the level may be dropped to reduce unnecessary information.
  \item For any types $A$ and $B$, we may form a type $A \to B$ that represents a function. And a term $f : A \to B$ of this type may be constructed by demonstrating that $f(a) : B$ under the assumption $a : A$.
  \item A dependent type is a morphism to the universe $P : A \to U$, so a dependent type $P : A \to U$ provides us with a type $P(a)$ for any $a : A$. A judgement $x : A \vdash B(x) : U$ is a dependent type over $A$.
\end{enumerate}

Now let's turn to several of the constructions more closely related to the homotopy theoretic side of things. Fibrations can be understood as a homotopy theoretic generalization of the notion of a fiber bundle. If working with spaces, a fibration is a map $\phi : E \to B$, which possesses a certain homotopy-lifting property. In this particular case we would refer to $B$ as the \textit{base space} and to $E$ as the \textit{total space} of the fibration. Given a point $b : B$, the \textit{fiber} over $b$ is just preimage $\phi^{-1}(b)$ of $b$ under the map $\phi$. The idea behind the fibration that it can be completely recovered from its base space $B$ together with its fibers by ``connecting'' the fibers together in accordance with the structure of the base space. Let's proceed with a more formal definitions\autocite{Warren1}.
The homotopy fiber of a morphism $f : E \to B$ over a point of $B$ is the collection of elements of $E$ that are mapped by $f$ to this point, hence it is the following pullback (or fiber product) of $pt$ ($\star$-valued point of $B$) and $f$:

\[
\begin{diagram}
	\node{E \times \star}
		\arrow{s,t}{}
		\arrow{e,t}{}
	\node{\star} 
		\arrow{s,r}{pt} \\
	\node{E}
		\arrow{e,r}{f} 
	\node{B}
\end{diagram}
\]

Lets call fibration any continuous map $p : E \to B$, which has the homotopy-lifting property from arbitrary spaces. $E$ will be called the total space and $B$ the base space of the fibration. There are a lot of different fibrations, which come with restriction to this general definition (e.g. trivial fibration, Serre fibration, and so on)\autocite{Warren1}.

A fibration $\phi$ is trivial if it is equivalent to a product fibration\autocite{Warren1}.

\[
\begin{diagram}
	\node{E}
		\arrow{s,t}{\phi}
		\arrow{e,t}{f}
	\node{B' \times F} 
		\arrow{s,r}{proj_{B'}} \\
	\node{B}
		\arrow{e,r}{\hat{f}} 
	\node{B'}
\end{diagram}
\]
    
Two fibrations over a circle with fiber given by the unit interval $[0,1]$ are the cylinder and the Moebius strip, they are visualized in Figure \ref{fig:1}.

\begin{figure}[H]
\centering
\begin{tikzpicture}[scale=1.25]
\begin{axis}[
    hide axis,
    view={40}{40}
]
\addplot3 [
    surf, shader = faceted interp,
    point meta = x,
    colormap/greenyellow,
    samples = 40,
    samples y = 5,
    z buffer = sort,
    domain = 0:360,
    y domain = -2:2
] (
    {sin(x)},
    {cos(x)},
    {y-20}
  );
  
\addplot3 [
    samples = 50,
    domain = 0:360,
    samples y = 0,
    thick
] (
    {cos(x)},
    {sin(x)},
    {-20}
  );   
  
\addplot3 [
    surf, shader = faceted interp,
    point meta = x,
    colormap/greenyellow,
    samples = 40,
    samples y = 5,
    z buffer = sort,
    domain = 0:360,
    y domain = -1:1
] (
    {(1+0.5*y*cos(x/2)))*cos(x)},
    {(1+0.5*y*cos(x/2)))*sin(x)},
    {0.5*y*sin(x/2)}
  );
  
\addplot3 [
    samples = 50,
    domain = -145:180,
    samples y = 0,
    thick
] (
    {cos(x)},
    {sin(x)},
    {0}
  ); 
\end{axis}
\end{tikzpicture}
\caption{Fibrations over a circle} \label{fig:1}
\end{figure}

\section{Moebius strip as an example of a fibration}

% анонс кода, код, пояснение кода (хотя бы чуть-чуть!), общее рассуждение, новый анонс и т.д. 

Now lets start with an example of combining fibrations and dependent types---formalizing Moebius strip using Arend proof-assistant.
Intuitively, we need to find a way to encode twist that distinguish Moebius strip from cylinder. Since the fiber is the interval, our goal is to ``rotate'' it. In order to do so we can get a path induced by a map that rotates interval using univalence. 
 
\begin{ListingEnv}[H]
\begin{lstlisting}
\func neg_neg : \Pi (x : I) -> ((inv seg) @ (inv seg @ x)) = x 
  => \lam x 
    =>  ((\lam x 
      => (\lam i 
        => inv (connAnd (inv seg) i)) (inv seg @ x)) x)
        # (inv ((\lam x 
          => ((\lam x 
            => (\lam i 
              => inv (connAnd seg i)) (seg @ x)) x) # seg) x))

\func twist : I = I => Iso=>Path
    \new Iso I I neg neg neg_neg neg_neg
\end{lstlisting}
\end{ListingEnv}

Now we are able to ``continuously'' get a twisted interval for every term of type $S_1$, which represents a circle.

\begin{ListingEnv}[H]
\begin{lstlisting}
\func M : \Pi (x : S1) -> \Type => \lam x => \case x \with {
  | base => I
  | loop i => twist @ i
}
\end{lstlisting}
\end{ListingEnv}

Finally, we can obtain a formalization of the Moebius strip as a total space, which is a pair of a point on a circle and a corresponding fiber.

\begin{ListingEnv}[H]
\begin{lstlisting}
\record Moebius (p1 : S1) {
  \field f1 : M(p1)
}
\end{lstlisting}
\end{ListingEnv}

\section{Isomorphism between a fibration and a dependent type}

To prove a correspondence between the fibration and the dependent type we need to formalize these two notions, to provide mutually inverse functions between these constructions and to prove that homotopies between compositions of these two functions and identity functions exist. 
A formalization of a dependent type is straightforward---it is a function that maps elements of some type to types.

\begin{ListingEnv}[H]
\begin{lstlisting}
\func family (B : \Type) : \Type => B -> \Type
\end{lstlisting}
\end{ListingEnv}

Now let's define the total space of a fibration $B \to U$.

\begin{ListingEnv}[H]
\begin{lstlisting}
\func total (B : \Type) (F : family B) : \Type 
	=> \Sigma (x : B) (F x)
\end{lstlisting}
\end{ListingEnv}

Also we need to define a projection that will serve as the trivial fibration.

\begin{ListingEnv}[H]
\begin{lstlisting}
\func trivial (B : \Type) (F : family B) : total B F -> B 
	=> \lam (x : total B F) => x.1
\end{lstlisting}
\end{ListingEnv}

The homotopy fiber of $f : A \to B$ over $base : B$ can be defined as an element of $A$ together with a path from $f x$ to $base$.

\begin{ListingEnv}[H]
\begin{lstlisting}
\func fiber (A B : \Type) (f : A -> B) (base : B)
   => \Sigma (x : A) (f x = base)
\end{lstlisting}
\end{ListingEnv}

To construct an element of a given dependent type parameterized by some element $y : B$ from the fiber of trivial projection over the same element we can use a wrapper around an eliminator for the interval type. 
  
\begin{ListingEnv}[H]
\begin{lstlisting}
\func encode (B : \Type) (F : B -> \Type) (y : B) :
      fiber (total B F) B (trivial B F) y -> F y
   => \lam (x : fiber (total B F) B (trivial B F) y)
   => subst B F x.1.1 y x.2 x.1.2
\end{lstlisting}
\end{ListingEnv}

An inverse of the previous function can be defined be direct fiber construction. 

\begin{ListingEnv}[H]
\begin{lstlisting}
\func decode (B : \Type) (F : B -> \Type) (y : B) :
      F y -> fiber (total B F) B (trivial B F) y
   => \lam (x : F y) => ((y, x), path (\lam i => y))
\end{lstlisting}
\end{ListingEnv}

To prove that a composition of decode and encode in this particular order is just an identity map we can prove that application of some element $x$ to the composition equals the exactly same element. The proof is straightforward since the outer function has an extra computational rule because it is defined with coe.

\begin{ListingEnv}[H]
\begin{lstlisting}
\func decode->encode (B : \Type) (F : family B) (y : B) :
  \Pi (x : F y) -> (encode B F y (decode B F y x)) = x
  => \lam (x : F y)
    => path (\lam i
      => encode
              B
              F
              ((decode B F y x).2 @ i)
              (decode B F y x))
\end{lstlisting}
\end{ListingEnv}

\begin{tikzpicture}
\draw plot [smooth cycle] coordinates {(-0.83,-2.31)(-1.73,0.25)(-1.15,2.29)(1.07,1.60)(1.74,-1.93)(0.31,-2.62)} 
node at (0.25,-3) {$F\ y$};

\draw plot [smooth cycle] coordinates {(5.83,-2.31)(3.73,0.25)(3.15,2.29)(6.07,1.60)(6.74,-1.93)} 
node at (5.65,-3) {$F\ x.1.1$};

\coordinate (p1) at (-0.5,-1);
\node[above left=5pt of {p1}, outer sep=2pt] {$encode\ B\ F\ y\ x$};
\filldraw (p1) circle[radius=1.5pt];

\coordinate (p2) at (5.75,-1.52);
\node[above right=5pt of {p2}, outer sep=2pt] {$coe\ A\ a\ right$};
\filldraw (p2) circle[radius=1.5pt];

\coordinate (p3) at (4.22,1.44);
\node[above=5pt of {p3}, outer sep=2pt] {$x.1.2$};
\filldraw (p3) circle[radius=1.5pt];

\draw[-latex] (p1) to [bend left] node [sloped,midway,below=0.5cm]{$path\ (\\lam\ i\ =>\ coe\ A\ a\ i)$} (p2);

\draw[-latex] [densely dotted](p2) to [bend right] node [sloped,midway,above]{?} (p3);

\draw[-latex][densely dotted] (p1) to[bend left] node [sloped,midway,above right=0cm and 0.3cm]{$p$} (p3);
\end{tikzpicture}

\begin{ListingEnv}[H]
\begin{lstlisting}
\func encode->decode (B : \Type) (F : family B) (y : B) :
  \Pi (x : fiber (total B F) B (trivial B F) y)
    -> (decode B F y (encode B F y x)) = x
  => \lam (x : fiber (total B F) B (trivial B F) y)
    => path (\lam i
      => (((inv x.2) @ i,
           (pathOverFamily (transport_twist F x.2 x.1.2)) @ i),
          (pathOverFamily ((coe_path (inv x.2) rfl rfl)
                      # (comp-assoc (inv (inv x.2)) rfl rfl)
                         # (refl-right ((inv (inv x.2)) # rfl))
                            # (refl-right (inv (inv x.2)))
                               # (inv_inv x.2))) @ i))
          \where
            \func transport_twist {A : \Type} (B : A -> \Type)
                                  {a b : A}
                                  (p : a = b) (x : B a) :
              transport B (inv p) (transport B p x) = x
              => J (\lam z (p' : a = z)
                => transport B (inv p') (transport B p' x) = x) rfl p
\end{lstlisting}
\end{ListingEnv}

To obtain an element of the identity type $(fiber\ (total\ B\ F)\ B\ (trivial\ B\ F)\ y)\ =\ (F\ y)$ from equivalence we can use an univalence. To do so we can use a wrapper around Arend built-in function iso that implies the univalence axiom.

\begin{ListingEnv}[H]
\begin{lstlisting}
\func hFiber=DependentTypeParameterized (B : \Type) 
	(F : family B) (y : B) : 
	(fiber (total B F) B (trivial B F) y) = (F y)
  => Iso=>Path 
  	(\new Iso 
  		(fiber (total B F) B (trivial B F) y) 
  		(F y) 
  		(encode B F y) 
  		(decode B F y) 
  		(encode->decode B F y) 
  		(decode->encode B F y))
\end{lstlisting}
\end{ListingEnv}

We can go even further. Since we have proved that for every element of base type $(fiber\ (total\ B\ F)\ B\ (trivial\ B\ F)\ y)\ =\ (F\ y)$ holds, we can prove that family itself equals a fiber over an arbitrary point.

\begin{ListingEnv}[H]
\begin{lstlisting}
\func Fibration=DependentType (B : \Type) (F : family B) : 
  (fiber (total B F) B (trivial B F)) = F
  => path (\lam i y
    => (hFiber=DependentTypeParameterized B F y) @ i)
\end{lstlisting}
\end{ListingEnv}

\Conc

In this paper I provided the proof of a correspondence between dependent types---basic type theoretic constructions, and fibrations---basic homotopy theoretic constructions in the proof-assistant Arend. This correspondence shows that working on homotopy theory can be approached syntactically if using type theory with ease, because there is no need for complex definitions of simplest homotopy construction.

\begin{otherlanguage}{english}
\printbibliography[%{}
    heading=bibintoc%
    ,title=Bibliography
]
\end{otherlanguage}

\end{document}
